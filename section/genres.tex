\section{Considering Genres based on Music Theory}\label{sec:genres}
	Musical factors contribute to the perceived music emotion.
	
	Major scale may be associated with happiness, joy, graceful, serene and solemn while minor scale may be associated with sadness, dreamy, dignified, tension, disgust and anger~\cite{JS11}. It suggests that major scale may be used to generate a happy melody.
	T-Music can generate melody in different scales in addition to ordinary major/minor scale.
	To generate melody with oriental style, major pentatonic scale should be used. It is obtained by omitting the 4\textsuperscript{th} (fa) and the 7\textsuperscript{th} (ti) notes in the corresponding major scale. The separation between the adjacent notes in the scale is $<$2, 2, 3, 2, 3$>$.  
	A video of composing melody in major pentatonic scale is available at \url{www.cse.ust.hk/~testing}.
		
	Fast tempo may induce excitement, happiness, potency, surprise, flippancy, anger and fear while slow tempo may induce calmness, peace, sadness, dignity, tenderness, longing, boredom and disgust~\cite{JS11}. 
	The tempo that equals or greater than 108 bpm is considered as fast.
	If a melody is played in a tempo that falls into ``Presto (168-208 bpm)'', it is considered as fast and likely to expresses happiness.
		
	Vocal range is the measure of the breath of pitches that a human voice can produce. High pitch ranges may express happy, graceful, serene, dreamy and exciting while low pitch ranges may express sadness, dignity, solemnity, vigour and excitement \cite{JS11}. 
	T-Music can shift the melody sentence by sentence so that the whole melody falls into the chosen vocal range such as ``Soprano (C4-C6)'', which is a high pitch range.
	
	Harmony refers to the ``vertical'' relationship between simultaneous pitches.
	Consonant harmony may represent happy, relaxed, graceful, serene, dreamy, dignified, serious and majestic while dissonant harmony may represent excitement, tension, vigour, anger, sadness and unpleasantness~\cite{JS11}.
	Intervals with half step equal to 1 (m2), 2 (M2), 6 (A4/d5), 10 (m7), 11 (M7) are considered to be dissonant intervals.
	For a happy song, the consonant intervals in the melody should be maximized.
	To achieve this, the retrieved ($pt$, $\tau$) tuples are sorted in descending order by the number of consonant intervals that the resulting melody based on $pt$ would have.
	
	Melody direction describes the tread of the notes. 
	Ascending melody may represent dignity, serenity, tension and happiness while descending melody may represent exciting, graceful, vigorous and sadness~\cite{JS11}. 
	A dignity melody should consist of more ascending treads. 
	To achieve this, the corresponding ($pt$, $\tau$) tuples are sorted in descending order by the vertical distance that $pt$ has ``climbed up''.
	
	Large pitch variation may be associated with happiness, pleasantness, activity and prise while small pitch variation may be associated with disgust, anger, fear and boredom~\cite{JS11}.\footnote{The association of ``Small pitch variation'' with anger and fear is controversial.}
	If we want to generate a melody with small pitch variations, ($pt$, $\tau$) is sorted in ascending order by the summation of the absolute value of the entries in $pt$. 
	
	Hevner's summary states how to use factors (including scale, tempo, vocal range, rhythm, harmony, melody direction) simultaneously for a specific perceived emotion~\cite{JS11}.\footnote{Since the definition of rhythm is not quantitative, it will not be used.} 
	For example, it states that emotion ``dignified-solemn'' should be in minor scale, slow tempo, low pitch range, firm rhythm, simple harmony and ascending melody.
	Besides, we can also mine rules from songs. For example, we can mine the distributions of major/minor, key signature, tempo range and vocal range from a song database and uses these distributions to compose melody that should be in similar style with the song database.	